\input{setup/preamble.tex}
\input{setup/macros.tex}

\begin{document}
    %%% Title %%%
    \tcbset{colframe=title,colback=title,arc=0mm}
    \begin{tcolorbox}

        \begin{minipage}{0.9\textwidth} % Name and Contact Info
            \vspace{.5em}

            % \name{Ian Soledade}{Innovative aerospace engineering student enrolled in the Georgia Tech BS/MS Honors Program with junior standing. Seeking to leverage STEM experience into private aerospace applications through a productive engineering internship.} % Name and Profession

            % \name{Ian Soledade}{Innovative aerospace engineering student enrolled in the Georgia Tech BS/MS Honors Program with junior standing. Seeking to further my research interests in applied fluid and plasma dynamics through an undergraduate research opportunity.} % Name and Profession

            \name{Ian Soledade}{Innovative aerospace engineering student enrolled in the Georgia Tech BS/MS Honors Program with senior standing. Seeking to further engineering team experience through an internship focused on providing intuitive real-world results.} % Name and Profession
            
            \vspace{1em}
            \email{isoledade3@gatech.edu} $\cdot$
            \phone{+1 (720) 755-5683} $\cdot$
            \github{https://github.com/OrangeKnight185}{GitHub}
            %$\cdot$
            %\address{Boulder, CO}
        \end{minipage}
        
    \end{tcolorbox}

    %%% Sections %%%
    \vspace*{-1em}
    \tcbset{colframe=white,colback=white,arc=0mm, height=0.8\textheight}
    \begin{tcolorbox}
        \vspace*{-0.5em}
        \begin{minipage}[t]{0.175\textwidth} % Side Panel (e.g. Skills, Links, Languages, etc.)
            \begin{tcolorbox}[height=0.85\textheight, grow to left by=0.55cm,colback=backdrop,colframe=backdrop,arc=0mm]

                % Skills, the skill level is drawn as bars, input: skill name and an array starting from 0 and ending before 4
                \subsection*{\Large{Skills}}
                    \skill{Math}{0, 1, 2, 3, 4}
                    \skill{Simulation}{0, 1, 2, 3, 4}
                    \skill{Data Processing}{0, 1, 2, 3}
                    \skill{Physics}{0, 1, 2}
                    \skill{Leadership}{0, 1, 2, 3}
                    \skill{Windows}{0, 1, 2, 3, 4}
                    \skill{UNIX}{0, 1, 2, 3, 4}
                    \skill{Solidworks}{0, 1, 2, 3}
                    \skill{MS Office}{0, 1, 2, 3}
                    \skill{VS Code}{0, 1, 2, 3}
                    \skill{Git}{0, 1, 2}
                    \skill{ParaView}{0, 1}
                    \skill{Tecplot}{0, 1, 2}

                \vspace{-1.15em}

                \subsection*{\Large{Languages}}
                    % \lan{Python}{0, 1, 2, 3, 4}
                    % \lan{C++}{0, 1, 2, 3}
                    % \lan{MATLAB}{0, 1, 2}
                    % \lan{LaTeX}{0, 1, 2}
                    Python \newline
                    C++ \newline
                    MATLAB \newline
                    \LaTeX \newline
                    C\# \newline
                    Java

                \vspace{-.4em}
                    
            \end{tcolorbox}
        \end{minipage}
        \begin{minipage}[t]{0.825\textwidth} % Main Panel (e.g. Education, Work Experience)
            \begin{tcolorbox}[grow to right by=0.95cm,height=0.85\textheight,colframe=white,colback=white]

                % Experience
                \section*{\Large{Experience}}


                    \work{\href{https://www.lockheedmartin.com/en-us/products/next-generation-interceptor.html}{Lockheed Martin Space, Engineering Intern}}{May 2024 - Aug 2024}{Deer Creek Site}{

                        \begin{itemize}

                            % \item Working controlled/classified aspects of the Next Generation Interceptor program. Focusing on meeting material and performance requirements alongside training in advanced concepts.
                        
                            \item Managed a project automating key validation tasks using Python/Excel, streamlining the electromechanical assembly verification process and pulling in deadlines, correlating directly to award fees.

                            \item Presented uncontrolled aspects of automation project to a virtual corporate tech forum for all of PMP engineering at Lockheed Martin Space.

                            \item Engaged in invitation-only lecture series about graduate level hypersonic fluid dynamics and heat transfer. Course explored non-equilibrium chemistry, radiation effects, and testing/simulation considerations.
                            
                        \end{itemize}}
                        
                    \vspace{1em}

                    \work{\href{https://ramblinrocketclub.org}{Ramblin' Rocket Club}}{Aug 2023 - May 2025}{Georgia Tech Student Organization}{

                        \begin{itemize}

                            \item Designed an N-class solid rocket motor from scratch. Devised, modeled, and manufactured a novel nozzle design utilizing zirconia \& aluminum with significant weight, impulse, and manufacturing benefits.

                            \item Worked as PR/Outreach lead for the NASA Student Launch Initiative competition team, handling STEM engagement of students within Georgia as well as sponsorship materials and social media.

                            \item Managed engineering competition for high school students to design and build electronic rocket payloads for data collection, including facilitation of preliminary and critical design reviews.

                            \item Received the NASA Artemis Pay It Forward award for the outreach impact of the Student Launch Initiative team through the competition and other significant educational aerospace activities.

                            \item Co-authored a paper outlining the competition and presented at the AIAA Region II Student Conference.
                            
                        \end{itemize}}

                    \vspace{1em}
                    
                    \work{\href{https://ccl.gatech.edu}{Computational Combustion Laboratory}}{Aug 2024 - May 2025}{Georgia Institute of Technology}{

                        \begin{itemize}
                        
                            \item Created novel 3D visualizations and animations of flow patterns, vorticity effects, and shockwave propagation through rotating detonation engines (RDEs) using Tecplot visualization software.

                            \item Implemented an algorithm to track detonation and shock waves throughout RDEs during data visualization. Analyzed fuel spray data in relation to wave propagation to assess combustion effects/stability.

                            \item Data generated utilizing in-house simulation toolkit LESLIE running on Georgia Tech PACE server supercomputer infrastructure.
                            
                        \end{itemize}
                    
                    }

                    \vspace{1em}

                    

                    \work{\href{https://www.colorado.edu/lab/ngpdl/}{Hall-effect Thruster Research}}{Sep 2022 - Apr 2023}{University of Colorado at Boulder}{
                    
                        \begin{itemize}
    
                            \item Created analytical models for low-power (<9kW) Hall-effect thrusters (HET) using MATLAB. Achieved less than 1\% error compared to measured data from the H9 HET, aiding significant reductions in computation time compared to numerical models.
    
                            \item Authored an internal paper describing the modeling process, code, and results, available on \href{https://github.com/OrangeKnight185/HET-Research-Paper}{GitHub}.
    
                            \item Research guided by PhD student mentor at the Non-equilibrium Gas and Plasma Dynamics Laboratory, NASA textbook materials, and research papers.
                            
                        \end{itemize}}

                    \vspace{1em}

                     
                    
                    % \work{\href{https://rocketry.gatech.edu/divisions/gtxr/}{Georgia Tech Experimental Rocketry}}{Aug 2023 - Mar 2024}{Ramblin' Rocket Club, Georgia Tech}{

                    %     \begin{itemize}

                    %         \item Competed in a design contest for an N-class rocket motor, utilizing solid fuel. Devised a novel hybrid nozzle constructed out of zirconia and aluminum with significant performance/manufacturing benefits.
                        
                    %         \item Collaborated in design discussions with the propulsion sub-team, contributed to motor and test stand design, as well as assisted in DRMs for several projects.

                    %         \item Completed courses towards becoming a certified machinist through the campus Aerospace Machine Shop, working towards independent precision operation of mill, lathe, and waterjet tools.
                            
                    %     \end{itemize}}

                   

                    
                
                    % \work{\href{https://materovcompetition.org}{MATE Underwater Robotics Competition}}{Sep 2021 - Jun 2023}{Robosharks, Boulder High Robotics}{
                    
                    %     \begin{itemize}
    
                    %         \item Electrical and software engineering team captain. Placed \href{https://20693798.fs1.hubspotusercontent-na1.net/hubfs/20693798/2022/2022_World_master_scoresheet_FINAL_RANGER.pdf}{6/25 in the 2022 world finals}, and \href{https://20693798.fs1.hubspotusercontent-na1.net/hubfs/20693798/2023\%20World\%20Championship\%20Scores\%20-\%20RANGER.pdf}{10/31 in the 2023 world finals}, receiving the first place award for our engineering presentation in 2023.
    
                    %         % \item Provided mentorship for 40+ club members. Organized smaller teams to solve problems, share knowledge, and create teaching opportunities. Encouraged members to take on leadership roles to develop our abilities as a club. 
    
                    %         \item \href{https://github.com/OrangeKnight185/2023-PCB-Files}{Designed a custom PCB with}: two micro-controllers, a 9 DoF IMU interface, Ethernet, sensor I/O, 12 servo controls, and power conditioning.
    
                    %         \item Developed algorithms to stabilize the underwater vehicle using IMU data to automate 8 thruster outputs for 6-DoF attitude control and to generate 3D environmental mesh structures from camera data.
    
                    %         % \item \href{https://github.com/OrangeKnight185/2023-Ranger-Code}{Coded all software functions:} initialization of sensors/thrusters, Ethernet packet construction and communication, thruster/manipulator arm/camera tilt control, desktop software for piloting, and data logging. %Additionally, \href{https://github.com/OrangeKnight185/2023-Ranger-Code}{created a desktop application} for a pilot to interface with the ROV.

                    %         % \item \href{https://github.com/OrangeKnight185/2023-WiFi-Float}{Designed a profiling float} to simulate data collection and transmission in the ocean with a buoy. %The float contained two Arduino microcontrollers, the first controlling the buoyancy engine, and the second hosting a WebSocket server via an encrypted WiFi network.
                            
                    %     \end{itemize}}

                    
                \vspace{-2.15em}

                % Education
                \section*{\Large{Education}}

                    \education{\href{https://ae.gatech.edu/bsms-honors-program}{In-progress BS, Aerospace Engineering}}{Aug 2023 - Present}{Georgia Institute of Technology}{GPA: 3.9}{\begin{itemize}

                        \item Undertaking a study abroad program in Limerick, Ireland focused on advanced coursework in hypersonic fluid dynamics and thermal analysis, jet propulsion systems, and control theory.

                        \item Completed coursework in CAD/CAM design, fluid and solid dynamics, advanced mathematics, physics, circuitry, chemistry, MSE, scripting and simulation, deformable bodies, and more.

                        \item Accepted into the BS/MS honors program, on track to receive MS degree within a year of BS graduation.

                        \item Pursuing a business minor in leadership studies, learning valuable leadership and collaboration skills. 

                        \item Involved in multiple on-campus organizations such as AIAA, GT Taido (Karate), Cloudman chess club, GT Dungeons and Dragons club, and a nationwide political organization.

                        
                    \end{itemize}}

                    \vspace{1em}

            \end{tcolorbox}
        \end{minipage}
    \end{tcolorbox}
    
\end{document}